\documentclass[xcolor={usenames,x11names}]{beamer}

\usepackage{default}
\usepackage{lmodern}
\usepackage{booktabs}
\usepackage{attrib}
\usepackage{listings}


\usetheme{Boadilla}
\usecolortheme{whale}

\title[What is UNIX?]{System Architecture}
\author{Joseph Booker}
\institute[UToledo]{Department of Physics and Astronomy \\ University of Toledo}

\setbeamertemplate{section page}
{%
    \begin{centering}
    \begin{beamercolorbox}[sep=12pt,center]{part title}
    \usebeamerfont{section title}\insertsection\par
    \end{beamercolorbox}
    \end{centering}
}

\begin{document}

\frame{\titlepage}

\begin{frame}{Layout of Today}
        \begin{itemize}
                \item What is a compiled program?
                \pause\item Communicating with hardware (aka: what is a kernel?)
                \pause\item Other program attributes: who owns them?
                \pause\item File permissions
                \pause\item Structure of File System
        \end{itemize}
\end{frame}

\begin{frame}[fragile]
\frametitle{What is a Program?}
 \begin{itemize}
  \item Code which the computer understands
  \pause\item In Linux/UNIX, program use the system to do the hard work.\begin{lstlisting}[language={[x86masm]Assembler},basicstyle=\tiny]
.data
msg: .ascii "Hello World\n"

.text
.global _start

_start:
movq $1, %rax   ; use the write syscall
movq $1, %rdi   ; write to stdout
movq $msg, %rsi ; use string "Hello World"
movq $12, %rdx  ; write 12 characters
syscall         ; make syscall

movq $60, %rax  ; use the _exit syscall
movq $0, %rdi   ; error code 0
syscall         ; make syscall
\end{lstlisting}
 \end{itemize}
\end{frame}

\begin{frame}{Overview of Kernel}\centering
 \includegraphics[scale=0.55]{kernelovr-arch}
\end{frame}

\begin{frame}{More Properties of Linux/UNIX Programs: User-owned}\centering
 \includegraphics[scale=0.5]{multiuse}
\end{frame}

\begin{frame}[fragile]
\frametitle{UNIX Permissions}
 \begin{itemize}
  \item All files have have an ``owning'' user and group.
  \pause\item Files can have Read, Write, and Execute permissions based on matching username or group.
  \pause\item These permissions are meaningful for directories:\begin{itemize}
                                                       \item Anyone with write permission can add/delete/rename files
                                                       \item Anyone with read permission can list files
                                                       \item Anyone with execute permission can \verb|cd| into this directory
                                                      \end{itemize}
 \end{itemize}
\end{frame}

\begin{frame}{Describing Permissions of a File --- ls -l listing}
  \centering\includegraphics[width=\textwidth]{permissions_diagram.png}
\end{frame}

\begin{frame}{Setting permissions of a file --- chmod syntax}
\begin{center}
 \begin{tabular}{lr}
  \toprule
  \textbf{Permission} & \textbf{Code} \\
  \midrule
  Execute & 1 \\ \midrule
  Write & 2 \\ \midrule
  Read & 4 \\
  \bottomrule
 \end{tabular}
\end{center}
\begin{itemize}
 \pause\item Example 1: Permission to read and write is $2+4 = 6$.
 \pause\item Example 2: Permission to read and execute is $1 + 4 = 5$.
 \pause\item Permissions are given as a three digit number for the owning user, group, and everyone else.\pause\ E.g., read/write permission for owner and read permission for everyone else would be $644$.
\end{itemize}
\end{frame}

\begin{frame}[fragile]
\frametitle{File System Structure}
	\begin{tabular}{@{}ll@{}}
		\pause\verb|/bin/|, \verb|/usr/bin/|, \verb|/sbin/|, \verb|/usr/sbin| & Programs\\
		\midrule
		\pause\verb|/Applications/| & Mac OS X graphical programs\\
		\midrule
		\pause\verb|/dev/| & Hardware Access\\
		\midrule
		\pause\verb|/etc/| & System-wide settings\\
		\midrule
		\pause\verb|/home/|, \verb|/Users/| & Home directories\\
		\midrule
		\pause\verb|/lib|, \verb|/lib64|, \verb|/usr/lib|, \verb|/Libraries| & Code used by programs\\
		\midrule
		\pause\verb|/tmp/| & Temporary storage for programs\\
		\midrule
		\pause\verb|/var/| & Permanent storage for programs\\
		\midrule
		\pause\verb|/usr/local/| & Third-party software\\
		\midrule
		\pause\verb|/usr/| & Originally for non-essential\\
		&software\\
		\bottomrule
	\end{tabular}
\end{frame}

\begin{frame}{Further Resources}
 \begin{itemize}
  \item Interactive tool for understanding commands --- \url{http://explainshell.com/}
  \item Software Carpentry Lesson --- \url{https://swcarpentry.github.io/shell-novice/}
  \item Command Line Crash Course --- \url{http://cli.learncodethehardway.org/book/}
 \end{itemize}
\end{frame}
 
\end{document}
